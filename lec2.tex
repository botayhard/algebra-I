\documentclass[a4paper,12pt]{article}

%%% Работа с русским языком
\usepackage{cmap}					% поиск в PDF
\usepackage{mathtext} 				% русские буквы в фомулах
\usepackage[T2A]{fontenc}			% кодировка
\usepackage[utf8]{inputenc}			% кодировка исходного текста
\usepackage[english,russian]{babel}	% локализация и переносы

%%% Дополнительная работа с математикой
\usepackage{amsfonts,amssymb,amsthm,mathtools} % AMS
\theoremstyle{plain}
\usepackage{amsmath}
\usepackage{icomma} % "Умная" запятая: $0,2$ --- число, $0, 2$ --- перечисление


% Номера формул
\mathtoolsset{showonlyrefs=true} % Показывать номера только у тех формул, на которые есть \eqref{} в тексте.

%% Шрифты
\usepackage{euscript}	 % Шрифт Евклид
\usepackage{mathrsfs} % Красивый матшрифт
\usepackage{soul} % выделение теорем и определений в разрядку

%% Свои команды
\DeclareMathOperator{\sgn}{\mathop{sgn}}

%% Перенос знаков в формулах (по Львовскому)
\newcommand*{\hm}[1]{#1\nobreak\discretionary{}
	{\hbox{$\mathsurround=0pt #1$}}{}}

%%% Работа с картинками
\usepackage{graphicx}  % Для вставки рисунков
\graphicspath{{images/}{images2/}}  % папки с картинками
\setlength\fboxsep{3pt} % Отступ рамки \fbox{} от рисунка
\setlength\fboxrule{1pt} % Толщина линий рамки \fbox{}
\usepackage{wrapfig} % Обтекание рисунков и таблиц текстом

%%% Работа с таблицами
\usepackage{array,tabularx,tabulary,booktabs} % Дополнительная работа с таблицами
\usepackage{longtable}  % Длинные таблицы
\usepackage{multirow} % Слияние строк в таблице


\title{НМУ. Алегбра-I}
\author{Максим Изумрудов}
\date{2017}
\usepackage[left=2cm,right=2cm,top=2cm,bottom=2cm]{geometry}

\begin{document}
\newtheorem*{def*}{\underline{Опр}}
\newtheorem*{def'}{\underline{Опр`}}
\begin{def*}
$Поле\; это\; множество\; K\; с\; операциями\; +\; и\; \cdot :\; K\times K\rightarrow K,\; такими\; что$

$1.\; \forall a,b,c \in K \;\;\; a+(b+c)=(a+b)+c\;(ассоциативность\; сложения);$

$2.\; \exists 0 \in K\; \forall a \in K \;\;\; a+0=0+a=0\;(наличие нуля);$

$3.\; \forall \in K \; \exists (-a) \in K\;\;\; a+(-a)=(-a)+a=0\;(наличие\; противоположного);$

$4.\; \forall a,b \in K\;\;\; a+b=b+a\;(коммутативность\; сложения);$

$5.\; \forall a,b,c \in K \;\;\; (a+b)\cdot c=a\cdot c+b\cdot c;\;c\cdot (a+b)=c\cdot a+c\cdot b\;(дистрибутивность\; умножения\\относительно\; сложения);$

$6.\; \forall a,b,c \in K \;\;\; a\cdot(b\cdot c)=(a\cdot b)\cdot c\;(ассоциативность\; умножения);$

$7.\; \exists 1 \neq 0 \; \forall a \in K \;\;\; a\cdot 1=1\cdot a=a\;(наличие\; единицы);$

$8.\; \forall a\neq 0 \in K \; \exists a^{-1}\in K \;\;\; a\cdot a^{-1}=a^{-1}\cdot a=1\;(наличие\; обратного);$

$9.\; \forall a,b\in K\;\;\; a\cdot b=b\cdot a \;(коммутативность\; умножения).$
\end{def*}

\begin{def*}
$G\; с\; операцией\; +:\; G\times G \rightarrow G\,; удовлетвлетворяющий\; 1., 2., 3.,\; называется\; \underline{группой}.$
\end{def*}

\begin{def*}
$Если\; выполнены\; 1-7,\; то\; K\; называется\; \underline{кольцом}.$
\end{def*}

\begin{def*}
$Если\; выполнены\; 1-7\; и\; 9,\; то\; K\; называется\; \underline{коммутативным\; кольцом}.$
\end{def*}

\begin{def*}
$Если\; выполнены\; 1-8,\; то\; К\; называется\; \underline{телом}.$
\end{def*}

\underline{\textbf{Примеры}}:

$\mathbb{N}$ - ничего

$\mathbb{Z}$ - коммутативное кольцо

$\mathbb{Q, R, Z}$ - поля

$\mathbb{Z}[$x$], \mathbb{R}[$x$], \mathbb{C}[$x$]$ - коммутативное кольцо

$K - поле \Rightarrow K[x]$ - коммутативное кольцо

$m \in \mathbb{N} \;\; множество\; остатков\; от\; деления\; на\; m\; \{0, 1, 2, ..., m-1\}\;-\;комутативное\; кольцо\\ (обозначается \mathbb{Z}/_{m}\mathbb{Z})$

$B_{m}=\{f: \; \{0, 1\}^{n} \rightarrow \{0, 1\}\}$ - коммутативное кольцо(Булева алгебра)

$(Пример)f(x_{1, ..., x_{n}})=x_{1}\; и\; (x_{2}\; или\; x_{3})\; и\; (не\; x_{4})$

$f_{1}+f_{2} := f_{1} XOR f_{2}=f_{1}+f_{2}(mod 2)$

\begin{tabular}{c|c|c}
XOR&0&1\\ \cline{1-3}
0&0&1\\ \cline{1-3}
1&1&0
\end{tabular}

$f_{1} \cdot f_{2} := f_{1} \; и \; f_{2}=f_{1}\cdot f_{2}(mod 2)$

$X\; -\; множество \;\;\; \wp(X)\; -\; множество\; подмножеств\; X\;\;\; коммутативное\; кольцо$

$A, B \subset X$

$A+B=A\bigtriangleup B=(A\setminus B) \cup (B\setminus A)$

$AB=A\cap B$

$R\; -\; коммутативное\; кольцо$

\underline{\textbf{Замечание}}:

$\bullet\; -a\; единственный$

$\bullet\; 1, a_{-1}\; единственные$

$\bullet\; a-b\; -\; решение\; x+b=a$

$\bullet\; a-(b-c)=a-b+c$

$\bullet\; a\cdot(-b)=-(a\cdot b)$

$\bullet\; a\setminus b\; -\; это\; x,\; что\; x\cdot b=a$ 

$\bullet\; \frac{a}{b\cdot c}=(\frac{a}{b}):c$

$\bullet\; a\cdot 0=0=0\cdot a$

\begin{def*}
$a\vdots b,\; если\; \exists с:\; a=b\cdot b$
\end{def*}

\underline{\textbf{Свойства}} $a\vdots c, b\vdots c\;\Rightarrow\; a+b, a-b\;\vdots\; c$

$a\vdots c\; \Rightarrow \; ab\vdots c$

$a\vdots b, c\vdots d\;\Rightarrow\; ac\vdots bd$

$a\vdots b, b\vdots c\;\Rightarrow\; a\vdots c$

\underline{\textbf{Замечание}} \boxed{ac\vdots bc\;\Rightarrow\; a\vdots b, c\neq 0} -- неверно если $\exists дел\; 0$ 

$ac=bc\cdot d\;\Rightarrow\; a=b\;\cdot d$

$(a-bd)\cdot c=0$

$\;\;\;\;\;\;\;\;\;\;\;\;\;\;\;\nparallel$

$\;\;\;\;\;\;\;\;\;\;\;\;\;\;\;0$

\begin{def*}
$x\; -\; \underline{делитель\; нуля},\; если\; x\neq 0\; \; \exists y\neq 0, такие\; что\; xy=0$
\end{def*}

$Пример:\; R=\mathbb{Z}/_{6}\mathbb{Z},\; x=2\; x=3\; -\; делители\; нуля$

\underline{\textbf{Замечание}} $Если\; в\; R\; нет\; делителей\; 0,\; то\; из\; ac=bc\; и\; c\neq 0 \; следует\; a=b$

\underline{\textbf{Замечание}} $R=\mathbb{Z}/_{6}\mathbb{Z}$

$x=3$

$x^{2}=3=x$

$R\; -\; комутативное\; кольцо\; без\; делителей\; нуля.$

\begin{def*}
$Кольцо\; без\; делителей\; нуля\; называется\; \underline{целостным}\; (или\; областью)$
\end{def*}

\begin{def*}
$элемент\; x\in R\; \underline{обратим},\; если\; \exists y\in R:\; xy=1\; (если\; 1\vdots x\in R).\\ Множество\; обратимых\; элементов\;обозначается\; R^{*}=\{x\;|\;x\; обр\; в\; R\} $
\end{def*}

\underline{\textbf{Примеры}}:

$R=\mathbb{Z}\;\;\; \mathbb{Z}^{*}=\{1, -1\}$

$K\; -\; поле\;\;\; R=K[x]\;\;\; R^{*}=\{c\in K, c\neq 0\}$

\underline{\textbf{Замечание}} $R^{*}\; -\; группа\; по\; умножению$

$x, y\; -\; обратимые,\; xy\; тоже\; обратим$

$1=x\cdot u$

$1=y\cdot v$

$1=(xy)(uv)$

$\underline{Эквивалентность}\\ X\; -\; множество$

$\underline{Отношение}\; на\; X\; -\; это\; подмножество\; S\subset X\times X,\; если\; (x, y) \in S\; пишут\; xSy$

$<:\; \{(x, y)\; |\;x<y\},\; \equiv mod 3 \{(x, y)\;|\;|x-y|:3\}$

\begin{def*}
$S\; -\; отношение\; эквивалентности,\; если$

$\bullet\; \forall x\;\;\; (x, x)\in S\;\;\;рефлексивность$

$\bullet\; (x, y)\in S\; \Rightarrow\; (y,x)\in S\;\;\;симмитричность$

$\bullet\; (x,y)\in S,\; (y, z)\in S\; \Rightarrow\; (x,z)\in S\;\;\;транзитивность$

$x \in X\;\; [x]=\{y\;|\;xSy\}=\{y\;|\;x\sim y\}\;\;\textendash \;\; класс\;эквивалентности$

$S\; обозначается\; \sim$
\end{def*}

$Множество\; классов\; эквивалентности\; X\; по\; отношению\; \sim\; обозначается\; X/_{\sim}\\(Фактор\; множество\; по\; отношению\; \sim)$

\begin{def*}
$x\; эквивалентен\; y,\; если x\vdots y,\; y\vdots x$
\end{def*}
\begin{def'}
$x\; эквивалентен\; y,\; если x=y\cdot c,\; c - обратимый$
\end{def'}

\underline{\textbf{Утверждение}} $Опр\; \sim\; Опр`$

\boxed{\Leftarrow} $\;\;\; x=y\cdot c\;\Rightarrow\;x\vdots y$

$\;\;\;\;\;\;\;\;\;\; c^{-1}=y\;\Rightarrow\;y\vdots\;x$

\boxed{\Rightarrow} $\;\;\; x=y\cdot u$
$\;\;\;\;\;\;\;\;\;\; y=x\cdot v$

$\;\;\;\;\;\;\;\;\;\; x=x\cdot v\cdot u$

$\;\;\;\;\;\;\;\;\;\; x(1-vu)=0$

$Если\; x\neq0,\; то\; 1=vu\;\Rightarrow\; v, u\;-\; обратимы\;\Rightarrow\;Опр`$

\begin{def*}
$x\in R\;\; x\;-\;неприводим,\; если\; x\neq 0,\; x \; - \; необратим\; x=ab\;\; a, b\in R,\; следует,\; что\; a\; или\\b\; обратим$
\end{def*}

\begin{def*}
$a, b\in R\;\;\; НОД(a, b)\;-\; это d\in R: a\vdots d,\; b\vdots d\; и\; \forall d$'$\;верно\; d\vdots d$'
\end{def*}

$Существование\; это\; открытый\; вопрос$

$Единственность$:

$d_{1}\; и\; d_{2}\;\; НОД(a, b),\; то\; d_{1}\sim d_{2}$

\textbf{Пример}$\; R=\mathbb{R}[x,y]$

$\;\;\;\;\;\;\;\;\;\;\;\;\;\;\;\;a=x$

$\;\;\;\;\;\;\;\;\;\;\;\;\;\;\;\;b=y$

\textbf{"Теорема" о существовании и единственности разложения на неприводимые}

$\forall x\in R\;\;\; x\neq\;\; x\; необратимый\;\; \exists x=p_{1}\cdot\dots\cdot p_{n}\;\; p_{i}\;-\;неприводим$

$если x=p_{1}\cdot\dots\cdot p_{n}=q_{1}\cdot\dots\cdot q_{m}\;\; p_{i}, q_{j}\; неприводимы,\; то\; n=m\; и\; с\; точностью\; до\; перестановки\\ сомножителей\; p_{i}\sim q_{i}\; для\; всех\; i$

$\underbrace{\;\;\;\;\;\;\;\;\;\;\;\;\;\;\;\;\;\;\;\;\;\;\;\;\;\;\;\;\;\;\;\;\;\;\;\;\;\;\;\;\;\;\;\;\;\;\;\;\;\;\;\;\;\;\;\;\;\;\;\;\;\;\;\;\;\;\;\;\;\;\;\;\;\;\;\;\;\;\;\;\;\;\;\;\;\;\;\;\;\;\;\;\;\;\;\;\;\;\;\;\;\;\;\;\;\;\;\;\;\;\;\;\;\;\;\;\;\;\;\;\;\;\;\;\;\;\;\;\;\;\;\;\;\;\;\;\;\;\;\;\;\;\;\;}$

$\;\;\;\;\;\;\;\;\;\;\;\;\;\;\;\;\;\;\;\;\;\;\;\;\;\;\;\;\;\;\;\;\;\;\;\;\;\;\;\;\;\;\;\;\;\;\;\;\;\;Вообще\; говоря\; неверно!$

\textbf{примеры} $\; R=\mathbb{Z}\;-\; верно$

$\;\;\;\;\;\;\;\;\;\;\;\;\;\;\;\;\;\; K\;-\; поле\;\;R=K[x]\;-\; верно$

$\;\;\;\;\;\;\;\;\;\;\;\;\;\;\;\;\;\; K[x_{1}, x_{2}, \dots, x_{n}]\;\; коэффициенты\; из\; поля\; -\; верно$

$\;\;\;\;\;\;\;\;\;\;\;\;\;\;\;\;\;\; \mathbb{Z}[x_{1}, x_{2}, \dots, x_{n}]\; -\; верно$

$R=\{P\in \mathbb{R}[x]\; |\; P\;\; a_{0}+a_{2}x^{2}+a_{3}x^3{}\}$

$\;\;\;\;\;\;\;\;\;\;\;\;\;\;\;\;\;\;\;\;\;\;\;\;\;\;\;\;\;\;\;\;\;\;\;\;\;\cap$

$\;\;\;\;\;\;\;\;\;\;\;\;\;\;\;\;\;\;\;\;\;\;\;\;\;\;\;\;\;\;\;\;\;\;\mathbb{R}[x]$

$x^{2}, x^{3}\;-\;неприводимы$

$X^{6}=x^{2}\cdot x^{2}\cdot x^{2}=x^{3}\cdot x^{3}$

\begin{def*}
$Кольцо\;R\;-\;факториально,\; если\; в\; R\; верна\; ОТА_{р}$
\end{def*}

\begin{def*}
$f:\; R_{1} \rightarrow R_{2} \;\;(R_{1}, R_{2}\; -\; кольца)\;\;\; f\;-\; гомоморфизм,\; если\;$

$\bullet\; f(a+b)=f(a)+f(b)$

$\bullet\; f(ab)=f(a)f(b)$

$\bullet\; f(1)=1$
\end{def*}


\underline{\textbf{Свойства}}:

$f(0)=0$

$f(-a)=-f(a)$

$f(a^{-1})=f^{-1}(a)\;\;(если\;\;существует)$

\textbf{Примеры}

$\mathbb{Z}\subset\mathbb{Q}\subset\mathbb{R}\subset\mathbb{C}:\; вложение\;-\;гомоморфизм$

$\mathbb{Z}\leftarrow\mathbb{Z}/_{m}\mathbb{Z}$

$\mathbb{C} \rightarrow \mathbb{C}\;\;\;\; z\mapsto \overline{z}$

$R \longrightarrow Frac R\;-\; поле\; частных$

$FracR=\{\frac{a}{b}\;|\;a, b\in R,\; b\neq 0\}$

$отношение\; эквивалентности\;\; \frac{a}{b}\sim\frac{c}{d},\; если\; ad=bc$

$[\frac{a}{b}]+[\frac{c}{d}]=[\frac{ad+bc}{bd}];\;\; [\frac{a}{b}]\cdot[\frac{c}{d}]:=[\frac{ab}{cd}]$

$\;\parallel$

$[\frac{a'}{b'}]$
\end{document}